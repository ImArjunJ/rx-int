\documentclass[conference]{IEEEtran}
\usepackage{lipsum}
\usepackage{graphicx}
\usepackage{cite}
\usepackage{amsmath,amssymb}
\usepackage{hyperref}

\begin{document}

\title{RX-INT: Kernel and Hypervisor-Level Instrumentation for Stealthy Memory-Resident Payload Detection and Extraction}

\author{%
    \IEEEauthorblockN{Arjun Juneja}
    \IEEEauthorblockA{School of Electronics and Computer Science\\
    University of Southampton\\
    Southampton, United Kingdom\\
    aj2g24@soton.ac.uk}
}

\maketitle

\begin{abstract}
RX-INT, a kernel-level driver for reliably detecting and dumping manually mapped or reflective DLLs and shellcode in protected Windows processes. The approach uses documented and undocumented kernel callbacks (thread tracking, VAD traversal), user-mode ETW telemetry, and an optional hypervisor-based trap to capture executable pages before they self-destruct or evade traditional detection. I implement and evaluate RX-INT in a Windows 11 environment, achieving relatively low overhead and high coverage against common anti-reverse-engineering techniques.
\end{abstract}

\begin{IEEEkeywords}
Windows security, manual mapping, kernel driver, memory forensics, VAD, ETW, hypervisor, reverse engineering
\end{IEEEkeywords}

\bibliographystyle{IEEEtran}
\bibliography{references}

\end{document}
